%\listfiles                                                                                                                                                     
\documentclass[12pt]{article}
\usepackage[usenames,dvipsnames]{xcolor}
\usepackage[utf8]{inputenc}
\usepackage[normalem]{ulem} % for strikeout \sout{...}                                                                                                          
\usepackage{amsmath, amsfonts, amssymb}
\usepackage{comment}

\usepackage{array}
\usepackage{times}
\usepackage{latexsym}
\usepackage{hyperref}
\hypersetup{backref,
%pdfpagemode=FullScreen,                                                                                                                                        
colorlinks=true,
linkcolor=red,
filecolor=red,
citecolor=blue}
\usepackage{epsfig}
\usepackage{bm}% bold math      

\newcommand{\ptjtk}{\rho(\vec{t}_j\cdot\vec{t}_k)}     
\newcommand{\tjtk}{(\vec{t}_j\cdot\vec{t}_k)}                                                                                                      
                                                                                                 

\title{Title}
\author{Author}
\date{}

%%% BEGIN DOCUMENT                                                                                                                                              
\begin{document}

\maketitle
\tableofcontents

\section{}

Given a signal $\vec{s}$ with parameters $\vec\theta$:
   \begin{equation}
   \vec\theta = \{m_1,m_2,\chi,\rho\}
   \end{equation}
where $m_1$ and $m_2$ are the masses, $\chi$ is the effective spin, and $\rho$ is the nominal signal-to-noise-ratio.

Probability that the signal $\vec{s}$ is recovered in a chirp mass bin $i$:
   \begin{equation}
   P(\text{bin $i$} | \vec{s}(\vec\theta)) = \sum_{\text{$k$ templates in bin $i$}} P(\text{$\vec{s}$ recovered by $\vec{t}_k$}|\vec{s}(\vec\theta))
   \end{equation}
where $P(\text{$\vec{s}$ recovered by $\vec{t}_k$}|\vec{s}(\vec\theta))$ is the probability that the signal is recovered by some template $\vec{t}_k$. $\vec{t}_k$ is the $k$th template that lies on the unit sphere, and $|\vec{t}_k| = 1$.

   \begin{align}
   \vec{d} &= \vec{n} + \vec{s} \\
   \rho_{obs,k}\vec{t}_k &= \vec{n} + \rho\vec{t}_j
   \end{align}

Here, we assume that that signal $\vec{s}$ is described by one of the templates in the bank. $\rho_{obs,k}$ is the observed signal-to-noise ratio.

Isolating for $\vec{n}$, we can solve for the squared magnitude $|\vec{n}|^2$:

   \begin{align}
   \vec{n} &= \rho_{obs,k}\vec{t}_k - \rho\vec{t}_j \\
   |\vec{n}|^2 &= (\rho_{obs,k}\vec{t}_k - \rho\vec{t}_j) \cdot (\rho_{obs,k}\vec{t}_k - \rho\vec{t}_j)\\
               &= \rho^2 + \rho_{obs,k}^2 - 2\rho_{obs,k}\ptjtk
   \label{eqn:n_squaredmagnitude}
   \end{align}
   
$\vec{n}$ is Gaussian distributed with constant variance of 1, so the probability density function for $N$ dimensions is given as

   \begin{equation}
   f(|\vec{n}|) = \frac{1}{(2\pi)^{N/2}}e^{-\frac{1}{2}|\vec{n}|^2}
   \end{equation}

We want to find the probability that $\vec{d}=\rho\vec{t}_j+\vec{n}$ lies inside the conic volume of some template $\vec{t}_k$ with solid angle $\Delta\Omega$. If it is inside the conic volume, it means that it is recoverable by that template $\vec{t}_k$.

   \begin{equation}
   P(\text{$\vec{s}$ recovered by $\vec{t}_k$}|\vec{s}(\vec\theta)) = \frac{\int_0^\infty \frac{1}{(2\pi)^{N/2}} e^{-\frac{1}{2} |\vec{n}|^2} {\rho^{N-1}_{obs,k}} d\rho_{obs,k} \Delta\Omega}
   {\sum_{\vec{t}_k} \int_0^\infty \frac{1}{(2\pi)^{N/2}} e^{-\frac{1}{2} |\vec{n}|^2} {\rho^{N-1}_{obs,k}} d\rho_{obs,k} \Delta\Omega}
   \label{eqn:prob_s_recovered_by_tk}
   \end{equation}

Let's deal with the numerator in Equation \ref{eqn:prob_s_recovered_by_tk} first. Moving constant terms outside the integral and substituting in Equation \ref{eqn:n_squaredmagnitude}, we get

   \begin{align}
   &= \frac{\Delta\Omega}{(2\pi)^{N/2}} \int_0^\infty  e^{-\frac{1}{2}( \rho^2 + \rho_{obs,k}^2 - 2\rho_{obs,k}\ptjtk)  } {\rho^{N-1}_{obs,k}} d\rho_{obs,k} \\
  &= \frac{\Delta\Omega}{(2\pi)^{N/2}} e^{-\frac{1}{2}\rho^2(1-(\vec{t}_j\cdot\vec{t}_k)^2)} \int_0^\infty  e^{-\frac{1}{2} (\rho_{obs,k} - \ptjtk)^2  } {\rho^{N-1}_{obs,k}} d\rho_{obs,k}
   \end{align}
Let $x = \rho_{obs,k} - \ptjtk$,

   \begin{align}
   &= \frac{\Delta\Omega}{(2\pi)^{N/2}} e^{-\frac{1}{2}\rho^2(1-(\vec{t}_j\cdot\vec{t}_k)^2)} \int_{-\ptjtk}^\infty  e^{-\frac{1}{2} x^2  } {(x+\ptjtk)^{N-1}} dx \\
   &= \frac{\Delta\Omega}{(2\pi)^{N/2}} e^{-\frac{1}{2}\rho^2(1-(\vec{t}_j\cdot\vec{t}_k)^2)} \int_{-\ptjtk}^\infty  e^{-\frac{1}{2} x^2  } {\sum_{n=0}^{N-1} \binom{N-1}{n}x^n(\ptjtk)^{N-1-n} } dx \\
   &= \frac{\Delta\Omega(\ptjtk)^{N-1}}{(2\pi)^{N/2}  e^{\frac{1}{2}\rho^2(1-(\vec{t}_j\cdot\vec{t}_k)^2)}}  \sum_{n=0}^{N-1} \binom{N-1}{n} (\ptjtk)^{-n} \int_{-\ptjtk}^\infty e^{-\frac{1}{2} x^2  } x^n  dx
   \end{align}

We can do the same for the denominator, and cancel out the constants $\Delta\Omega$, $\rho^{N-1}$, and $(2\pi)^{-N/2}$. And so Equation \ref{eqn:prob_s_recovered_by_tk} becomes the following:
   
   \begin{equation}
   =  \frac{ \tjtk^{N-1}e^{-\frac{1}{2}\rho^2(1-\tjtk^2)} \sum_{n=0}^{N-1} \binom{N-1}{n} (\ptjtk)^{-n} \int^{\infty}_{-\ptjtk} e^{-\frac{1}{2}x^2}x^n dx} { \sum_{\vec{t}_k} \tjtk^{N-1}e^{-\frac{1}{2}\rho^2(1-\tjtk^2)} \sum_{n=0}^{N-1} \binom{N-1}{n} (\ptjtk)^{-n} \int^{\infty}_{-\ptjtk} e^{-\frac{1}{2}x^2}x^n dx}
   \end{equation}

We argue that for very large $N$, the dominant term in the denominator is in the case $\vec{t}_k = \vec{t}_j$, such that $\tjtk = 1$. We also know the indefinite integral $\int e^{ax^r}x^{b-1} = -(1/r) x^b (-ax^r)^{-b/r} \Gamma(b/r,-ax^r)$ (link \href{http://functions.wolfram.com/ElementaryFunctions/Exp/21/01/02/01/01/08/0001/}{here}). Therefore, this becomes:

   \begin{align}
   &\approx  \frac{ \tjtk^{N-1}e^{-\frac{1}{2}\rho^2(1-\tjtk^2)} \sum_{n=0}^{N-1} \binom{N-1}{n} (\ptjtk)^{-n} \int^{\infty}_{\ptjtk} e^{-\frac{1}{2}x^2}x^n dx} 
   { (1)^{N-1}e^{-\frac{1}{2}\rho^2(1-1^2)} \sum_{n=0}^{N-1} \binom{N-1}{n} (\rho(1))^{-n} \int^{\infty}_{\rho(1)} e^{-\frac{1}{2}x^2}x^n dx} \\
   &= \frac{ \tjtk^{N-1}e^{-\frac{1}{2}\rho^2(1-\tjtk^2)} \sum_{n=0}^{N-1} \binom{N-1}{n} (\ptjtk)^{-n} \int^{\infty}_{\ptjtk} e^{-\frac{1}{2}x^2}x^n dx} 
   { e^0\sum_{n=0}^{N-1} \binom{N-1}{n} \rho^{-n} \int^{\infty}_{\rho} e^{-\frac{1}{2}x^2}x^n dx} \\
   &= \tjtk^{N-1}e^{-\frac{1}{2}\rho^2(1-\tjtk^2)} \frac{ \sum_{n=0}^{N-1} \binom{N-1}{n} (\ptjtk)^{-n} \Big[ -(2)^{\frac{1}{2}(n-1)}\Gamma\Big(\frac{n+1}{2},\frac{1}{2}x^2\Big) \Big]^{\infty}_{\ptjtk} } 
   { \sum_{n=0}^{N-1} \binom{N-1}{n} \rho^{-n}  \Big[ -(2)^{\frac{1}{2}(n-1)}\Gamma\Big(\frac{n+1}{2},\frac{1}{2}x^2\Big) \Big]^{\infty}_{\rho} } \\
   &= \tjtk^{N-1}e^{-\frac{1}{2}\rho^2(1-\tjtk^2)} \frac{ \sum_{n=0}^{N-1} \binom{N-1}{n} (\ptjtk)^{-n} \Big[ 2^{\frac{n}{2}}\Gamma\Big(\frac{n+1}{2},\frac{1}{2}(\ptjtk)^2\Big) \Big]} 
   { \sum_{n=0}^{N-1} \binom{N-1}{n} \rho^{-n}  \Big[ 2^{\frac{n}{2}}\Gamma\Big(\frac{n+1}{2},\frac{1}{2}\rho^2\Big) \Big] }
   \end{align}
   
We can expand $\binom{N-1}{n}$ in terms of Gamma functions: $\binom{N-1}{n} = \frac{(N-1)!}{n!(N-n-1)!} = \frac{\Gamma(N)}{\Gamma(n+1)\Gamma(N-n)}$ to get:% $(N-1)!$ appears in both numerator and denominator and can be canceled out. We also canceled out $2^{-\frac{1}{2}}$.

   \begin{align}
   &= \frac{\tjtk^{N-1}}{e^{\frac{1}{2}\rho^2(1-\tjtk^2)}} \frac{ \sum_{n=0}^{N-1} \frac{\Gamma(N)}{\Gamma(n+1)\Gamma(N-n)} (\ptjtk)^{-n} \Big[ 2^{\frac{n}{2}}\Gamma\Big(\frac{n+1}{2},\frac{1}{2}(\ptjtk)^2\Big) \Big]} 
   { \sum_{n=0}^{N-1} \frac{\Gamma(N)}{\Gamma(n+1)\Gamma(N-n)} \rho^{-n}  \Big[ 2^{\frac{n}{2}}\Gamma\Big(\frac{n+1}{2},\frac{1}{2}\rho^2\Big) \Big] } \\
   &= \frac{\tjtk^{N-1}}{e^{\frac{1}{2}\rho^2(1-\tjtk^2)}} \frac{ \sum_{n=0}^{N-1} \frac{\Gamma(N)}{\Gamma(n+1)\Gamma(N-n)} \Big(\frac{\sqrt{2}}{\ptjtk}\Big)^{n} \Gamma\Big(\frac{n+1}{2},\frac{1}{2}(\ptjtk)^2\Big) } 
   { \sum_{n=0}^{N-1} \frac{\Gamma(N)}{\Gamma(n+1)\Gamma(N-n)} \Big(\frac{\sqrt{2}}{\rho}\Big)^{n} \Gamma\Big(\frac{n+1}{2},\frac{1}{2}\rho^2\Big) } \\
   \end{align}
  
In the limit of $n \rightarrow \infty$, $\Gamma(a,x) \rightarrow \Gamma(a)$. This approximation is good enough when $a > 50$. Therefore:
   
   \begin{align}
   &\approx \frac{\tjtk^{N-1}}{e^{\frac{1}{2}\rho^2(1-\tjtk^2)}} \frac{ \sum_{n=0}^{N-1} \frac{\Gamma(N)}{\Gamma(n+1)\Gamma(N-n)} \Big(\frac{\sqrt{2}}{\ptjtk}\Big)^{n} \Gamma\Big(\frac{n+1}{2}\Big) } 
   { \sum_{n=0}^{N-1} \frac{\Gamma(N)}{\Gamma(n+1)\Gamma(N-n)} \Big(\frac{\sqrt{2}}{\rho}\Big)^{n} \Gamma\Big(\frac{n+1}{2}\Big) } \\
   &= \frac{\tjtk^{N-1}}{e^{\frac{1}{2}\rho^2(1-\tjtk^2)}} \frac{ \sum_{n=0}^{N-1} \frac{\Gamma(N)}{\Gamma(n+1)\Gamma(N-n)} \Big(\frac{\sqrt{2}}{\ptjtk}\Big)^{n} \Gamma\Big(\frac{n+1}{2}\Big) } 
   { \sum_{n=0}^{N-1} \frac{\Gamma(N)}{\Gamma(n+1)\Gamma(N-n)} \Big(\frac{\sqrt{2}}{\rho}\Big)^{n} \Gamma\Big(\frac{n+1}{2}\Big) }
   \end{align}
   
Rather than compute the entire summation term in both the numerator and denominator, we want to find the values of $n$ which contribute the most to the summation term. We cannot compute this directly because Python gives overflow errors. First, we can locate the peak of the individual summation term by locating the peak of the logarithm of the individual summation term.

   \begin{align}
   &= \frac{\tjtk^{N-1}}{e^{\frac{1}{2}\rho^2(1-\tjtk^2)}} 
   \frac{ \sum_{n=0}^{N-1} X_n } 
   { \sum_{n=0}^{N-1} Y_n } \\
   &= \frac{\tjtk^{N-1}}{e^{\frac{1}{2}\rho^2(1-\tjtk^2)}} 
   \frac{ \sum_{n=0}^{N-1} [\exp(\ln(X_0)),\exp(\ln(X_1)),...,\exp(\ln(X_{N-1}))] } 
   { \sum_{n=0}^{N-1} [\exp(\ln(Y_0)),\exp(\ln(Y_1)),...,\exp(\ln(Y_{N-1}))] }
   \end{align}

Let $\max[...] = \ln(X_\alpha), \ln(Y_\beta)$.

   \begin{align}
   &\rightarrow \frac{[\ln(X_0),\ln(X_1),...,\ln(X_{N-1})]}{[\ln(Y_0),\ln(Y_1),...,\ln(Y_{N-1})]} \\
   &= \frac{[\ln(X_0)-\ln(X_\alpha),\ln(X_1)-\ln(X_\alpha),...,\ln(X_{N-1})-\ln(X_\alpha)]}{[\ln(Y_0)-\ln(Y_\beta),\ln(Y_1)-\ln(Y_\beta),...,\ln(Y_{N-1})-\ln(Y_\beta)]} \\
   &= \frac{[\ln(X_0/X_\alpha),\ln(X_1/X_\alpha),...,\ln(X_{N-1}/X_\alpha))]}{[\ln(Y_0/Y_\beta),\ln(Y_1/Y_\beta),...,\ln(Y_{N-1}/Y_\beta)]} \\
   &\rightarrow \frac{[\exp(\ln(X_0/X_\alpha)),\exp(\ln(X_1/X_\alpha)),...,\exp(\ln(X_{N-1}/X_\alpha))]}{[\exp(\ln(Y_0/Y_\beta)),\exp(\ln(Y_1/Y_\beta)),...,\exp(\ln(Y_{N-1}/Y_\beta))]} \\
   &= \frac{[X_0/X_\alpha,X_1/X_\alpha,...,X_{N-1}/X_\alpha]}{[Y_0/Y_\beta,Y_1/Y_\beta,...,Y_{N-1}/Y_\beta]} \\
   &=\frac{\frac{1}{X_\alpha}[X_0,X_1,...,X_{N-1}]}{\frac{1}{Y_\beta}[Y_0,Y_1,...,Y_{N-1}]} \times \frac{X_\alpha}{Y_\beta}
   \end{align}

In the final equality, we multiply the term by $X_\alpha/Y_\beta$ to get back the correct value.

\begin{comment}
Looking at the denominator as an example, we can write
 
   \begin{align}
   \frac{d}{dN} &= \lim_{N_1\rightarrow N_2} \frac{ \sum_{n=0}^{N_2-1} \frac{1}{n!(N_2-n-1)!} \rho^{-n} 2^{n/2} \Gamma \Big(\frac{n+1}{2},\frac{\rho^2}{2}\Big) 
   - \sum_{n=0}^{N_1-1} \frac{1}{n!(N_1-n-1)!} \rho^{-n} 2^{n/2} \Gamma \Big(\frac{n+1}{2},\frac{\rho^2}{2}\Big) }
   {N_2-N_1} \\
   &\stackrel{N_1 = N_2-1}{=} \sum^{N_2-2}_{n=0} \frac{\rho^{-n}}{n!} \Big[2^{n/2}\Gamma \Big(\frac{n+1}{2},\frac{\rho^2}{2}\Big) \Big] \Big(\frac{1}{(N_2-n-1)!} - \frac{1}{(N_2-n-2)!} \Big) \\
   \notag & \qquad\qquad
   + \frac{1}{(N_2-1)!} \rho^{-(N_2-1)} \Big[ 2^{(N_2-1)/2} \Gamma \Big(\frac{N_2}{2},\frac{\rho^2}{2}\Big) \Big] \\
   &= \sum^{N_2-2}_{n=0} \frac{\rho^{-n}}{n!} \Big[2^{n/2}\Gamma \Big(\frac{n+1}{2},\frac{\rho^2}{2}\Big) \Big] \Big(\frac{1}{(N_2-n-1)!} - \frac{N_2-n-1}{(N_2-n-1)!} \Big) \\
   \notag & \qquad\qquad
   + \frac{1}{(N_2-1)!} \rho^{-(N_2-1)} \Big[ 2^{(N_2-1)/2} \Gamma \Big(\frac{N_2}{2},\frac{\rho^2}{2}\Big) \Big] \\
   &= \sum^{N_2-2}_{n=0} \frac{\rho^{-n}}{n!} \Big[2^{n/2}\Gamma \Big(\frac{n+1}{2},\frac{\rho^2}{2}\Big) \Big] \Big(-\frac{(N_2-n-2)}{(N_2-n-1)!} \Big) \\
   \notag & \qquad\qquad
   + \frac{1}{(N_2-1)!} \rho^{-(N_2-1)} \Big[ 2^{(N_2-1)/2} \Gamma \Big(\frac{N_2}{2},\frac{\rho^2}{2}\Big) \Big] \\
   &= \sum^{N_2-2}_{n=0} \frac{\rho^{-n}}{n!} \Big[2^{n/2}\Big(\frac{n}{2}-\frac{1}{2}\Big)! e^{-\rho^2/2} \sum_{m=0}^{n/2-1/2} \frac{(\rho^2/2)^m}{m!} \Big] \Big(-\frac{(N_2-n-2)}{(N_2-n-1)!} \Big) \\
   \notag & \qquad\qquad
   + \frac{1}{(N_2-1)!} \rho^{-(N_2-1)} \Big[ 2^{(N_2-1)/2} (N_2-1)! e^{-\rho^2/2} \sum_{m=0}^{N_2-1} \frac{(\rho^2/2)^m}{m!} \Big]
   \end{align}
\end{comment}
\begin{comment}   
We can show that the first term goes to zero {\bf NEED TO PROVE THIS}, which leaves us with the second term only. The same can be done for the numerator. Combining numerator and denominator together, we get:

   \begin{align}
   &= \tjtk^{N-1}e^{\frac{1}{2}\rho^2\tjtk^2} \frac{ \frac{1}{(N-1)!} (\ptjtk)^{-(N-1)} \Big[ 2^{(N-1)/2} \Gamma \Big(\frac{N}{2},\frac{(\ptjtk)^2}{2}\Big) \Big] } 
   { \frac{1}{(N-1)!} \rho^{-(N-1)} \Big[ 2^{(N-1)/2} \Gamma \Big(\frac{N}{2},\frac{\rho^2}{2}\Big) \Big] } \\
   &= e^{\frac{1}{2}\rho^2\tjtk^2} \frac{ \Gamma \Big(\frac{N}{2},\frac{(\ptjtk)^2}{2}\Big) } 
   { \Gamma \Big(\frac{N}{2},\frac{\rho^2}{2}\Big) } 
   \end{align}
\end{comment}   
   
   
Can use this special value of $\Gamma(n,z) = (n-1)!e^{-z}\sum^{n-1}_{m=0} \frac{z^m}{m!}$ (link \href{http://functions.wolfram.com/GammaBetaErf/Gamma2/03/01/02/0007/}{here} and \href{https://en.wikipedia.org/wiki/Incomplete_gamma_function#Special_values}{here}).


\begin{comment}
   \begin{align}
   &= e^{\frac{1}{2}\rho^2\tjtk^2} \frac{(\frac{N}{2}-1)!e^{-(\ptjtk)^2/2}\sum^{N/2-1}_{n=0} \Big(\frac{(\ptjtk)^2}{2}\Big)^{n}\frac{1}{n!} } 
   {(\frac{N}{2}-1)!e^{-(\rho)^2/2}\sum^{N/2-1}_{n=0} \Big(\frac{\rho^2}{2}\Big)^{n}\frac{1}{n!} } \\
   &= \lim_{N\rightarrow\infty} e^{\frac{1}{2}\rho^2\tjtk^2} e^{\frac{1}{2}\rho^2(1-\tjtk^2)}
   \frac{\sum^{N/2-1}_{n=0} \Big(\frac{(\ptjtk)^2}{2}\Big)^{n}\frac{1}{n!} } 
   {\sum^{N/2-1}_{n=0} \Big(\frac{\rho^2}{2}\Big)^{n}\frac{1}{n!} } \\
   &= e^{\frac{1}{2}\rho^2} \frac{e^{\frac{1}{2}(\ptjtk)^2}}{e^{\frac{1}{2}\rho^2}} \\
   & = e^{\frac{1}{2}(\ptjtk)^2} 
   \end{align}
\end{comment}

%%%%%%%%%%%%%%%%%%%%%%%%%%
%%%%%%%%%%%%%%%%%%%%%%%%%%

\begin{comment}
I want to find out at what $n$ is the factor $\binom{N-1}{n} (\tjtk)^{-n}$ maximised.

   \begin{align}
   y = \binom{N-1}{n}(\tjtk)^{-n} &= \frac{(N-1)!}{n!(N-1-n)!} (\tjtk)^{-n} \\
   y & \approx \frac{N!}{n!(N-n)!} (\tjtk)^{-n}
   \end{align}
since $N \gg 1$. If I further assume that $N$, $n$, and $N-n \ge 10^{5}$, I can use Stirling's approximation $\log n \approx n\log n - n$ and maintain 99.999\% accuracy (\href{http://www.luc.edu/faculty/dslavsk/courses/phys328/classnotes/Stirling.pdf}{link here}).

   \begin{align}
   \log y &= \log(N-1)! - \log n! - \log(N-1-n)! -n\log(\tjtk) \\
   &\approx \log N! - \log n! - \log(N-n)! - n\log(\tjtk) \\
   &\approx N\log N - N - n\log n + n \\ 
   &\notag \qquad - (N-n)\log(N-n) + (N-n)  -n\log\(\tjtk) \\
   &\approx N\log N - n\log n -(N-n)\log(N-n) -n\log(\tjtk)
   \end{align}

   \begin{align}
   \frac{d(\log y)}{dn} &= -\frac{n}{n} - \log n + \frac{N-n}{N-n} + \log(N-n) - \log (\tjtk)\\
   0 &= \log\Big\{\frac{(N-n)}{n\tjtk}\Big\}\\
   1 &= \frac{N-n}{n\tjtk}\\
   n &= \frac{N}{\tjtk+1}
   \end{align}
For Stirling's approximation to hold, I require:

   \begin{equation}
   \frac{10^5}{N-10^5} \le \tjtk \le \frac{N-10^5}{10^5}
   \end{equation}
For $N=10^9$, this gives $\frac{1}{9999} \le \tjtk \le 9999$, which is well within the expected values for $\tjtk$.

Therefore, the factor $\binom{N-1}{n} (\tjtk)^{-n}$ is maximised when $ n = {N}/({\tjtk+1})$. This gives:

   \begin{equation}
   \binom{N-1}{\frac{N}{\tjtk+1}} (\tjtk)^{-\frac{N}{\tjtk+1}} \approx \frac{N!}{\frac{N}{\tjtk+1}!\Big(N-\frac{N}{\tjtk+1}!\Big)} \tjtk^{-\frac{N}{\tjtk+1}}
   \end{equation}

Using Stirling's approximation $n! \approx \sqrt{2\pi n} (n/e)^n$:

   \begin{align}
   =& \Big\{ {2\pi N}^{1/2}{\Big(\frac{N}{e}\Big)}^N  \Big\} 
      \Big\{ {2\pi\frac{N}{\tjtk+1}}^{-\frac{1}{2}} \Big(\frac{N}{\tjtk+1}\frac{1}{e}\Big)^{-\frac{N}{\tjtk+1}}  \Big\} \\
   \notag & \qquad 
      \Big\{ {2\pi\frac{N\tjtk}{\tjtk+1}}^{-\frac{1}{2}} \Big(\frac{N\tjtk}{\tjtk+1}\frac{1}{e}\Big)^{-\frac{N\tjtk}{\tjtk+1}} \Big\} \tjtk^{-\frac{N}{\tjtk+1}}\\
   =& \frac{(\tjtk+1)^{N+1}}{\sqrt{2\pi N}} (\tjtk)^{-\frac{N\tjtk}{\tjtk+1}-\frac{1}{2}} (\tjtk)^{-\frac{N}{\tjtk+1}} \\
   =& \frac{(\tjtk+1)^{N+1}}{\sqrt{2\pi N}} (\tjtk)^{-\frac{1}{2}} (\tjtk)^{-\frac{N\tjtk -N}{\tjtk+1}}\\
   =& \frac{(\tjtk+1)^{N+1}}{\sqrt{2\pi N \tjtk}} (\tjtk)^{-N}\\
   \approx& \frac{(\tjtk+1)^{N}}{\tjtk} \frac{1}{\sqrt{2\pi N \tjtk}}
   \end{align}
Then:
   \begin{equation}
   \max\Big( \binom{N-1}{n} (\tjtk)^{-n}\Big) \approx \frac{(\tjtk+1)^N}{\sqrt{2\pi(\tjtk)^{3}N}}
   \end{equation}
\end{comment}

%%%%%%%%%%%%%%%%%%%%%%%%%%
%%%%%%%%%%%%%%%%%%%%%%%%%%

\end{document}
