%\listfiles                                                                                                                                                     
\documentclass[12pt]{article}
\usepackage[usenames,dvipsnames]{xcolor}
\usepackage[utf8]{inputenc}
\usepackage[normalem]{ulem} % for strikeout \sout{...}                                                                                                          
\usepackage{amsmath, amsfonts, amssymb}
\usepackage{comment}

\usepackage{array}
\usepackage{times}
\usepackage{latexsym}
\usepackage{hyperref}
\hypersetup{backref,
%pdfpagemode=FullScreen,                                                                                                                                        
colorlinks=true,
linkcolor=red,
filecolor=red,
citecolor=blue}
\usepackage{epsfig}
\usepackage{bm}% bold math      

\newcommand{\ptjtk}{\rho(\vec{t}_j\cdot\vec{t}_k)}     
\newcommand{\tjtk}{(\vec{t}_j\cdot\vec{t}_k)}                                                                                                      
                                                                                                 

\title{Title}
\author{Author}
\date{}

%%% BEGIN DOCUMENT                                                                                                                                              
\begin{document}

\maketitle
\tableofcontents

\section{}

Given a signal $\vec{s}$ with parameters $\vec\theta$:
   \begin{equation}
   \vec\theta = \{m_1,m_2,\chi,\rho\}
   \end{equation}
where $m_1$ and $m_2$ are the masses, $\chi$ is the effective spin, and $\rho$ is the nominal signal-to-noise-ratio.

Probability that the signal $\vec{s}$ is recovered in a chirp mass bin $i$:
   \begin{equation}
   P(\text{bin $i$} | \vec{s}(\vec\theta)) = \sum_{\text{$k$ templates in bin $i$}} P(\text{$\vec{s}$ recovered by $\vec{t}_k$}|\vec{s}(\vec\theta))
   \end{equation}
where $P(\text{$\vec{s}$ recovered by $\vec{t}_k$}|\vec{s}(\vec\theta))$ is the probability that the signal is recovered by some template $\vec{t}_k$. $\vec{t}_k$ is the $k$th template that lies on the unit sphere, and $|\vec{t}_k| = 1$.

   \begin{align}
   \vec{d} &= \vec{n} + \vec{s} \\
   \rho_{obs,k}\vec{t}_k &= \vec{n} + \rho\vec{t}_j
   \end{align}

Here, we assume that that signal $\vec{s}$ is described by one of the templates in the bank. $\rho_{obs,k}$ is the observed signal-to-noise ratio.

Isolating for $\vec{n}$, we can solve for the squared magnitude $|\vec{n}|^2$:

   \begin{align}
   \vec{n} &= \rho\vec{t}_j - \rho_{obs,k}\vec{t}_k \\
   |\vec{n}|^2 &= (\rho\vec{t}_j - \rho_{obs,k}\vec{t}_k) \cdot (\rho\vec{t}_j - \rho_{obs,k}\vec{t}_k)\\
               &= \rho^2 + \rho_{obs,k}^2 - 2\rho_{obs,k}\ptjtk
   \label{eqn:n_squaredmagnitude}
   \end{align}
   
$\vec{n}$ is Gaussian distributed with constant variance of 1, so the probability density function for $N$ dimensions is given as

   \begin{equation}
   f(|\vec{n}|) = \frac{1}{(2\pi)^{N/2}}e^{-\frac{1}{2}|\vec{n}|^2}
   \end{equation}

We want to find the probability that $\vec{d}=\rho\vec{t}_j+\vec{n}$ lies inside the conic volume of some template $\vec{t}_k$ with solid angle $\Delta\Omega$. If it is inside the conic volume, it means that it is recoverable by that template $\vec{t}_k$.

   \begin{equation}
   P(\text{$\vec{s}$ recovered by $\vec{t}_k$}|\vec{s}(\vec\theta)) = \frac{\int_0^\infty \frac{1}{(2\pi)^{N/2}} e^{\frac{1}{2} |\vec{n}|^2} {\rho^{N-1}_{obs,k}} d\rho_{obs,k} \Delta\Omega}{\sum_{\vec{t}_k} \int_0^\infty \frac{1}{(2\pi)^{N/2}} e^{\frac{1}{2} |\vec{n}|^2} {\rho^{N-1}_{obs,k}} d\rho_{obs,k} \Delta\Omega}
   \label{eqn:prob_s_recovered_by_tk}
   \end{equation}

Let's deal with the numerator in Equation \ref{eqn:prob_s_recovered_by_tk} first. Moving constant terms outside the integral and substituting in Equation \ref{eqn:n_squaredmagnitude}, we get

   \begin{align}
   &= \frac{\Delta\Omega}{(2\pi)^{N/2}} \int_0^\infty  e^{\frac{1}{2} \rho^2 + \rho_{obs,k}^2 - 2\rho_{obs,k}\ptjtk  } {\rho^{N-1}_{obs,k}} d\rho_{obs,k} \\
  &= \frac{\Delta\Omega}{(2\pi)^{N/2}} e^{-\frac{1}{2}\rho^2(1-(\vec{t}_j\cdot\vec{t}_k)^2)} \int_0^\infty  e^{-\frac{1}{2} (\rho_{obs,k} - \ptjtk)^2  } {\rho^{N-1}_{obs,k}} d\rho_{obs,k}
   \end{align}
Let $x = \rho_{obs,k} - \ptjtk$,

   \begin{align}
   &= \frac{\Delta\Omega}{(2\pi)^{N/2}} e^{-\frac{1}{2}\rho^2(1-(\vec{t}_j\cdot\vec{t}_k)^2)} \int_0^\infty  e^{-\frac{1}{2} x^2  } {(x+\ptjtk)^{N-1}} dx \\
   &= \frac{\Delta\Omega}{(2\pi)^{N/2}} e^{-\frac{1}{2}\rho^2(1-(\vec{t}_j\cdot\vec{t}_k)^2)} \int_{-\ptjtk}^\infty  e^{-\frac{1}{2} x^2  } {\sum_{n=0}^{N-1} \binom{N-1}{n}x^n(\ptjtk)^{N-1-n} } dx \\
   &= \frac{\Delta\Omega(\ptjtk)^{N-1}}{(2\pi)^{N/2}  e^{\frac{1}{2}\rho^2(1-(\vec{t}_j\cdot\vec{t}_k)^2)}}  \sum_{n=0}^{N-1} \binom{N-1}{n} (\ptjtk)^{-n} \int_{-\ptjtk}^\infty e^{-\frac{1}{2} x^2  } x^n  dx
   \end{align}

We can do the same for the denominator, and cancel out the constants. And so Equation \ref{eqn:prob_s_recovered_by_tk} becomes the following:
   
   \begin{equation}
   =  \frac{ \tjtk^{N-1}e^{-\frac{1}{2}\rho^2(1-\tjtk^2)} \sum_{n=0}^{N-1} \binom{N-1}{n} (\ptjtk)^{-n} \int^{\infty}_{\ptjtk} e^{-\frac{1}{2}x^2}x^n } { \sum_{\vec{t}_k} \tjtk^{N-1}e^{-\frac{1}{2}\rho^2(1-\tjtk^2)} \sum_{n=0}^{N-1} \binom{N-1}{n} (\ptjtk)^{-n} \int^{\infty}_{\ptjtk} e^{-\frac{1}{2}x^2}x^n }
   \end{equation}

We argue that for very large $N$, the dominant term in the denominator is in the case $\vec{t}_k = \vec{t}_j$, such that $\tjtk = 1$. We also know the indefinite integral $\int e^{ax^r}x^{b-1} = -(1/r) x^b (-ax^r)^{-b/r} \Gamma(b/r,-ax^r)$ (link \href{http://functions.wolfram.com/ElementaryFunctions/Exp/21/01/02/01/01/08/0001/}{here}). Therefore, Equation \ref{eqn:prob_s_recovered_by_tk} becomes:

   \begin{align}
   &=  \frac{ \tjtk^{N-1}e^{-\frac{1}{2}\rho^2(1-\tjtk^2)} \sum_{n=0}^{N-1} \binom{N-1}{n} (\ptjtk)^{-n} \int^{\infty}_{\ptjtk} e^{-\frac{1}{2}x^2}x^n } 
   { e^{-\frac{1}{2}\rho^2} \sum_{n=0}^{N-1} \binom{N-1}{n} \rho^{-n} \int^{\infty}_{\rho} e^{-\frac{1}{2}x^2}x^n } \\
   &= \frac{ \tjtk^{N-1}e^{\frac{1}{2}\rho^2\tjtk^2} \sum_{n=0}^{N-1} \binom{N-1}{n} (\ptjtk)^{-n} \int^{\infty}_{\ptjtk} e^{-\frac{1}{2}x^2}x^n } 
   { \sum_{n=0}^{N-1} \binom{N-1}{n} \rho^{-n} \int^{\infty}_{\rho} e^{-\frac{1}{2}x^2}x^n } \\
   &= \tjtk^{N-1}e^{\frac{1}{2}\rho^2\tjtk^2} \frac{ \sum_{n=0}^{N-1} \binom{N-1}{n} (\ptjtk)^{-n} \Big[ -(2)^{\frac{1}{2}(n-1)}\Gamma\Big(\frac{n+1}{2},\frac{1}{2}x^2\Big) \Big]^{\infty}_{\ptjtk} } 
   { \sum_{n=0}^{N-1} \binom{N-1}{n} \rho^{-n}  \Big[ -(2)^{\frac{1}{2}(n-1)}\Gamma\Big(\frac{n+1}{2},\frac{1}{2}x^2\Big) \Big]^{\infty}_{\rho} } \\
   &= \tjtk^{N-1}e^{\frac{1}{2}\rho^2\tjtk^2} \frac{ \sum_{n=0}^{N-1} \binom{N-1}{n} (\ptjtk)^{-n} 2^{\frac{n}{2}}\Gamma\Big(\frac{n+1}{2},\frac{1}{2}(\ptjtk)^2\Big) } 
   { \sum_{n=0}^{N-1} \binom{N-1}{n} \rho^{-n}  2^{\frac{n}{2}}\Gamma\Big(\frac{n+1}{2},\frac{1}{2}\rho^2\Big) }
   \end{align}
   
Looking at the denominator as an example, we can write
   
   \begin{align}
   \frac{d}{dN} &= \lim_{N_1\rightarrow N_2} \frac{ \sum_{n=0}^{N_2-1} \frac{1}{n!(N_2-n-1)!} \rho^{-n} 2^{n/2} \Gamma \Big(\frac{n+1}{2},\frac{\rho^2}{2}\Big) 
   - \sum_{n=0}^{N_1-1} \frac{1}{n!(N_1-n-1)!} \rho^{-n} 2^{n/2} \Gamma \Big(\frac{n+1}{2},\frac{\rho^2}{2}\Big) }
   {N_2-N_1} \\
   &\stackrel{N_1 = N_2-1}{=} \sum^{N_2-2}_{n=0} \frac{\rho^{-n}}{n!} \Big[2^{n/2}\Gamma \Big(\frac{n+1}{2},\frac{\rho^2}{2}\Big) \Big] \Big(\frac{1}{(N_2-n-1)!} - \frac{1}{(N_2-n-2)!} \Big) \\
   \notag & \qquad\qquad
   + \frac{1}{(N_2-1)!} \rho^{-(N_2-1)} \Big[ 2^{(N_2-1)/2} \Gamma \Big(\frac{N_2}{2},\frac{\rho^2}{2}\Big) \Big] \\
   &= \sum^{N_2-2}_{n=0} \frac{\rho^{-n}}{n!} \Big[2^{n/2}\Gamma \Big(\frac{n+1}{2},\frac{\rho^2}{2}\Big) \Big] \Big(\frac{1}{(N_2-n-1)!} - \frac{N_2-n-1}{(N_2-n-1)!} \Big) \\
   \notag & \qquad\qquad
   + \frac{1}{(N_2-1)!} \rho^{-(N_2-1)} \Big[ 2^{(N_2-1)/2} \Gamma \Big(\frac{N_2}{2},\frac{\rho^2}{2}\Big) \Big] \\
   &= \sum^{N_2-2}_{n=0} \frac{\rho^{-n}}{n!} \Big[2^{n/2}\Gamma \Big(\frac{n+1}{2},\frac{\rho^2}{2}\Big) \Big] \Big(-\frac{(N_2-n-2)}{(N_2-n-1)!} \Big) \\
   \notag & \qquad\qquad
   + \frac{1}{(N_2-1)!} \rho^{-(N_2-1)} \Big[ 2^{(N_2-1)/2} \Gamma \Big(\frac{N_2}{2},\frac{\rho^2}{2}\Big) \Big] \\
   \end{align}
   
We can show that the first term goes to zero {\bf NEED TO PROVE THIS}, which leaves us with the second term only. The same can be done for the numerator. Combining numerator and denominator together, we get:

   \begin{align}
   &= \tjtk^{N-1}e^{\frac{1}{2}\rho^2\tjtk^2} \frac{ \frac{1}{(N-1)!} (\ptjtk)^{-(N-1)} \Big[ 2^{(N-1)/2} \Gamma \Big(\frac{N}{2},\frac{(\ptjtk)^2}{2}\Big) \Big] } 
   { \frac{1}{(N-1)!} \rho^{-(N-1)} \Big[ 2^{(N-1)/2} \Gamma \Big(\frac{N}{2},\frac{\rho^2}{2}\Big) \Big] } \\
   &= e^{\frac{1}{2}\rho^2\tjtk^2} \frac{ \Gamma \Big(\frac{N}{2},\frac{(\ptjtk)^2}{2}\Big) } 
   { \Gamma \Big(\frac{N}{2},\frac{\rho^2}{2}\Big) } 
   \end{align}
   
Can use this special value of $\Gamma(n,z) = (n-1)!e^{-z}\sum^{n-1}_{m=0} \frac{z^m}{m!}$ (link \href{http://functions.wolfram.com/GammaBetaErf/Gamma2/03/01/02/0007/}{here} and \href{https://en.wikipedia.org/wiki/Incomplete_gamma_function#Special_values}{here}).

   \begin{align}
   &= e^{\frac{1}{2}\rho^2\tjtk^2} \frac{(\frac{N}{2}-1)!e^{-(\ptjtk)^2/2}\sum^{N/2-1}_{n=0} \Big(\frac{(\ptjtk)^2}{2}\Big)^{n}\frac{1}{n!} } 
   {(\frac{N}{2}-1)!e^{-(\rho)^2/2}\sum^{N/2-1}_{n=0} \Big(\frac{\rho^2}{2}\Big)^{n}\frac{1}{n!} } \\
   &= e^{\frac{1}{2}\rho^2\tjtk^2} e^{\frac{1}{2}\rho^2(1-\tjtk^2)}
   \frac{\sum^{N/2-1}_{n=0} \Big(\frac{(\ptjtk)^2}{2}\Big)^{n}\frac{1}{n!} } 
   {\sum^{N/2-1}_{n=0} \Big(\frac{\rho^2}{2}\Big)^{n}\frac{1}{n!} } \\
   &= \lim_{N\rightarrow\infty} e^{\frac{1}{2}\rho^2} \frac{e^{\frac{1}{2}(\ptjtk)^2}}{e^{\frac{1}{2}\rho^2}} \\
   & = e^{\frac{1}{2}(\ptjtk)^2} 
   \end{align}

%%%%%%%%%%%%%%%%%%%%%%%%%%
%%%%%%%%%%%%%%%%%%%%%%%%%%

\begin{comment}
I want to find out at what $n$ is the factor $\binom{N-1}{n} (\tjtk)^{-n}$ maximised.

   \begin{align}
   y = \binom{N-1}{n}(\tjtk)^{-n} &= \frac{(N-1)!}{n!(N-1-n)!} (\tjtk)^{-n} \\
   y & \approx \frac{N!}{n!(N-n)!} (\tjtk)^{-n}
   \end{align}
since $N \gg 1$. If I further assume that $N$, $n$, and $N-n \ge 10^{5}$, I can use Stirling's approximation $\log n \approx n\log n - n$ and maintain 99.999\% accuracy (\href{http://www.luc.edu/faculty/dslavsk/courses/phys328/classnotes/Stirling.pdf}{link here}).

   \begin{align}
   \log y &= \log(N-1)! - \log n! - \log(N-1-n)! -n\log(\tjtk) \\
   &\approx \log N! - \log n! - \log(N-n)! - n\log(\tjtk) \\
   &\approx N\log N - N - n\log n + n \\ 
   &\notag \qquad - (N-n)\log(N-n) + (N-n)  -n\log\(\tjtk) \\
   &\approx N\log N - n\log n -(N-n)\log(N-n) -n\log(\tjtk)
   \end{align}

   \begin{align}
   \frac{d(\log y)}{dn} &= -\frac{n}{n} - \log n + \frac{N-n}{N-n} + \log(N-n) - \log (\tjtk)\\
   0 &= \log\Big\{\frac{(N-n)}{n\tjtk}\Big\}\\
   1 &= \frac{N-n}{n\tjtk}\\
   n &= \frac{N}{\tjtk+1}
   \end{align}
For Stirling's approximation to hold, I require:

   \begin{equation}
   \frac{10^5}{N-10^5} \le \tjtk \le \frac{N-10^5}{10^5}
   \end{equation}
For $N=10^9$, this gives $\frac{1}{9999} \le \tjtk \le 9999$, which is well within the expected values for $\tjtk$.

Therefore, the factor $\binom{N-1}{n} (\tjtk)^{-n}$ is maximised when $ n = {N}/({\tjtk+1})$. This gives:

   \begin{equation}
   \binom{N-1}{\frac{N}{\tjtk+1}} (\tjtk)^{-\frac{N}{\tjtk+1}} \approx \frac{N!}{\frac{N}{\tjtk+1}!\Big(N-\frac{N}{\tjtk+1}!\Big)} \tjtk^{-\frac{N}{\tjtk+1}}
   \end{equation}

Using Stirling's approximation $n! \approx \sqrt{2\pi n} (n/e)^n$:

   \begin{align}
   =& \Big\{ {2\pi N}^{1/2}{\Big(\frac{N}{e}\Big)}^N  \Big\} 
      \Big\{ {2\pi\frac{N}{\tjtk+1}}^{-\frac{1}{2}} \Big(\frac{N}{\tjtk+1}\frac{1}{e}\Big)^{-\frac{N}{\tjtk+1}}  \Big\} \\
   \notag & \qquad 
      \Big\{ {2\pi\frac{N\tjtk}{\tjtk+1}}^{-\frac{1}{2}} \Big(\frac{N\tjtk}{\tjtk+1}\frac{1}{e}\Big)^{-\frac{N\tjtk}{\tjtk+1}} \Big\} \tjtk^{-\frac{N}{\tjtk+1}}\\
   =& \frac{(\tjtk+1)^{N+1}}{\sqrt{2\pi N}} (\tjtk)^{-\frac{N\tjtk}{\tjtk+1}-\frac{1}{2}} (\tjtk)^{-\frac{N}{\tjtk+1}} \\
   =& \frac{(\tjtk+1)^{N+1}}{\sqrt{2\pi N}} (\tjtk)^{-\frac{1}{2}} (\tjtk)^{-\frac{N\tjtk -N}{\tjtk+1}}\\
   =& \frac{(\tjtk+1)^{N+1}}{\sqrt{2\pi N \tjtk}} (\tjtk)^{-N}\\
   \approx& \frac{(\tjtk+1)^{N}}{\tjtk} \frac{1}{\sqrt{2\pi N \tjtk}}
   \end{align}
Then:
   \begin{equation}
   \max\Big( \binom{N-1}{n} (\tjtk)^{-n}\Big) \approx \frac{(\tjtk+1)^N}{\sqrt{2\pi(\tjtk)^{3}N}}
   \end{equation}
\end{comment}

%%%%%%%%%%%%%%%%%%%%%%%%%%
%%%%%%%%%%%%%%%%%%%%%%%%%%

\end{document}
